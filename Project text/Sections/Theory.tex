\section{Introductory Theory}

% TODO: Write an introduction to the theory section, whats the layout and what will be covered
% Talk about what parts are taken from BZ, and JZ, as well as what's original.

\import{./Theory_subsections/}{Basic_definitions.tex}
\import{./Theory_subsections/}{Symplectic_group.tex}
\import{./Theory_subsections/}{SymplecticStiefel.tex}

\section{Right-Invariant Framework on the \\Symplectic Stiefel Manifold}
One of the key insights of Bendokat and Zimmermann \cite[p.~11]{BendokatZimmermann2021} is that using a right invariant framework one is able to construct geodesics on $\mathrm{SpSt}(2n, 2k)$. In this section we will first define a right-invariant metric on $\mathrm{Sp}(2n)$ and its corresponding geodesics, then transport this metric to $\mathrm{SpSt}(2n, 2k)$. This will allow us to define geodesics on $\mathrm{SpSt}(2n, 2k)$. 

\import{./Theory_subsections/}{Right_invariant_framework.tex}
\import{./Theory_subsections/}{Geodesics.tex}
\import{./Theory_subsections/}{Riemannian_gradient.tex}
\import{./Theory_subsections/}{Riemannian_hessian.tex}
\import{./Theory_subsections/}{Theory_to_application.tex}
