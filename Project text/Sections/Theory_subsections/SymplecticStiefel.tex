\subsection{The Symplectic Stiefel manifold}
% Defining the basic structures on the symplectic Stiefel manifold
Now that we have defined the symplectic group, we introduce the manifold of interest, the real symplectic Stiefel manifold. It is defined as
\begin{equation}\label{eq:spst_def}
    \mathrm{SpSt}(2n, 2k)\coloneqq \{p\in \mathbb{R}^{2n\times2k} \;|\; p ^{\mathrm{T}}J_{2n}p=J_{2k}\}.
\end{equation}
Following \cite[Prop.~3.1]{BendokatZimmermann2021}, it is explicitly connected to the symplectic group in the sense that $\mathrm{SpSt}(2n, 2k)$ is diffeomorphic to the following quotient manifold of $\mathrm{Sp}(2n)$:
\begin{equation*}
    \mathrm{SpSt}(2n, 2k)\cong \mathrm{Sp}(2n)/\mathrm{Sp}(2(n-k)),
\end{equation*}
where the notion of quotient manifold is as in \Cref{def:quotient_manifold}. It has dimension $\mathrm{dim}(\mathrm{SpSt}(2n, 2k))=(2n-2k+1)k$. 

The \todo{Is a version of this info too much of a detour?}following piece of insight can give some further intuition on what the Symplectic Stiefel manifold is. We note that the Stiefel manifold is a quotient space, as defined in \Cref{def:quotient_manifold}, of the orthogonal group as defined in \Cref{def:orthogonal_group} such that $\mathrm{St}(2n, 2k)=\mathrm{O}(n)/\mathrm{O}(n-k)$. 

The expression for the tangent space follows straightforwardly from the definition of $\mathrm{SpSt}(2n, 2k)$. Assume we have a curve, $c(t)\in \mathrm{SpSt}(2n,2k)$, s.t. $c(0)=p$ and $\dot{c}(0):=\tfrac{d}{dt}c(t)|_{t=0}=X$. Since $c(t)$ is a curve in $\mathrm{SpSt}(2n,2k)$, by \eqref{eq:spst_def} it must satisfy the following condition:
\begin{equation}\label{eq:spst_tangent_space_derivation}
    c(t) ^{\mathrm{T}}J_{2n}c(t)=J_{2k}.
\end{equation}
Taking the derivative of \eqref{eq:spst_tangent_space_derivation} with respect to $t$ at $t=0$ we get
\begin{equation*}
    \dot{c}^{\mathrm{T}}(t)J_{2n}c(t)+c^{\mathrm{T}}(t)J_{2n}\dot{c}\big|_{t=0}=X ^{\mathrm{T}}J_{2n}p+p ^{\mathrm{T}}J_{2n}X=0_{2k}.
\end{equation*}
After moving the first term over to the left hand side, and multiplying with $J_{2n}$ from the left, we get
\begin{equation*}
    p^{+}X=-X^{+}p.
\end{equation*}
We recognize this condition as $p^{+}X\in \mathfrak{sp}(2k)$ as defined in \eqref{eq:sp_Lie_algebra}. This means that for a point $p$, 
\begin{equation}\label{eq:spst_tangent_space}
    T_{p}\mathrm{SpSt}(2n,2k)=\left\{X\in \mathbb{R}^{2n\times2k}\;\middle|\;p^{+}X\in\mathfrak{sp}(2k)\right\}.
\end{equation}
