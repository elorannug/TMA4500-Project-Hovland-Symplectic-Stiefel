\subsection{The Symplectic group}
The real \textit{symplectic group} is the space overarching the symplectic Stiefel manifold, \todo{Why do we need sp? We will use quotient properties to map stuff to Spst}and we will look at this space first. To be able to define the symplectic group we first need some preliminary definitions. Define the \textit{symplectic identity} as the following block matrix,
\begin{equation*}
    J_{2n}\coloneqq\begin{bmatrix}
        0_{n} & I_{n} \\
        -I_{n} & 0_{n}
    \end{bmatrix},
\end{equation*}
where $I_{n}$ denotes the $n\times n$ identity matrix. Note that throughout this report, if it is clear from the context, the subscript for $I$ and $J$ will be omitted for the sake of notational clarity.  $J_{2n}$ has some properties we will take advantage of frequently:
\begin{equation}\label{eq:J_2n_properties}
    J_{2n} ^{\mathrm{T}}=-J_{2n}=J_{2n}^{-1}
\end{equation}
The symplectic group is defined as the set of matrices which define the symplectic structure in the following sense. We define the real symplectic group as %is a quotient space of the general linear group (defined in Definition \ref{def:general_linear_group}). 
\begin{equation}\label{eq:sp_def}
    \mathrm{Sp}(2n)\coloneqq \{p\in \mathbb{R}^{2n\times2n} \;|\; p^{+}p=I_{2n}\},
\end{equation}
where $^{+}$ is defined as the \textit{symplectic inverse} of any matrix $q\in\mathbb{R}^{2n\times2k}$ such that 
\begin{equation}\label{eq:symplectic_inverse}
    q^{+}\coloneqq J_{2k}^{\mathrm{T}}q ^{\mathrm{T}}J_{2n}.
\end{equation}
The Lie algebra of $\mathrm{Sp}(2n)$ is the symplectic groups' tangent space at the identity. It is given by 
\begin{equation}\label{eq:sp_Lie_algebra}
    \mathfrak{sp}(2n)\coloneqq \{\Omega\in \mathbb{R}^{2n\times2n} \;|\; \Omega^{+}=-\Omega\},
\end{equation}
where $\Omega$ is called the Hamiltonian matrix \todo{Do I want to find a historical reference?}ref ??. 
Now we can define the tangent space of $\mathrm{Sp}(2n)$ at any point $p$ \todo{maybe explain a little more how we get this tangent space} as
\begin{equation}\label{eq:sp_tangent_space}
    T_{p}\mathrm{Sp}(2n)=\{\Omega p\in \mathbb{R}^{2n\times2n} \;|\; \Omega\in\mathfrak{sp}(2n)\}.
\end{equation}


