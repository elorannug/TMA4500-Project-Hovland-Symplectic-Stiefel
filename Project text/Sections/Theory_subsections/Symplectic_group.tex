\subsection{The Symplectic group}
The real \textit{symplectic group} is the space overarching the symplectic Stiefel manifold. In this section we will therefore define the necessary concepts for this group in preparation for the symplectic Stiefel manifold. Following \cite[p.~3]{BendokatZimmermann2021}, to be able to define the symplectic group we first need some preliminary definitions. Define the \textit{symplectic identity} as the following block matrix,
\begin{equation*}
    J_{2n}\coloneqq\begin{bmatrix}
        0_{n} & I_{n} \\
        -I_{n} & 0_{n}
    \end{bmatrix},
\end{equation*}
where $I_{n}$ denotes the $n\times n$ identity matrix. Note that throughout this report, if it is clear from the context, the subscript for $I$ and $J$ will be omitted for the sake of notational clarity.  $J_{2n}$ has some properties we will take advantage of frequently:
\begin{equation}\label{eq:J_2n_properties}
    J_{2n} ^{\mathrm{T}}=-J_{2n}=J_{2n}^{-1}
\end{equation}
The symplectic group is defined as the set of matrices which define the symplectic structure in the following sense. We define the real symplectic group as %is a quotient space of the general linear group (defined in Definition \ref{def:general_linear_group}). 
\begin{equation}\label{eq:sp_def}
    \mathrm{Sp}(2n)\coloneqq \{p\in \mathbb{R}^{2n\times2n} \;|\; p^{+}p=I_{2n}\},
\end{equation}
where $^{+}$ is defined as the \textit{symplectic inverse} of any matrix $q\in\mathbb{R}^{2n\times2k}$ such that 
\begin{equation}\label{eq:symplectic_inverse}
    q^{+}\coloneqq J_{2k}^{\mathrm{T}}q ^{\mathrm{T}}J_{2n}.
\end{equation}
Finding the tangent space of $\mathrm{Sp}(2n)$ follows  intuitively from the definition, which we will verify ourselves in the following derivation. Let us first look at the tangent space at the identity. Assume we have a curve $c_{I}(t)\in \mathrm{Sp}(2n)$, such that $c_{I}(0)=I$ and $\dot{c}_{I}(0)\coloneqq\tfrac{d}{dt}c_{I}(t)|_{t=0}=X$. Since $c_{I}(t)$ is a curve on $\mathrm{Sp}(2n)$, by \eqref{eq:sp_def} it must satisfy the following condition:
%
\begin{equation}\label{eq:sp_condition_derivation}
c_{I}(t)^{\mathrm{T}}J_{2n}c_{I}(t)=J_{2n}.
\end{equation}
%
Taking the derivative of \eqref{eq:sp_condition_derivation} with respect to $t$ at $t=0$, \eqref{eq:sp_condition_derivation} becomes 
%
\begin{equation}\label{eq:sp_condition_derivation_2}
\dot{c}_{I}(t)^{\mathrm{T}}Jc_{I}(0)+c_{I}(0)^{\mathrm{T}}J \dot{c}_{I}(0)=0.
\end{equation}
%
Recalling that $c_{I}(0)=I$, defining $\Omega \coloneqq \dot{c}_{I}(0)$, and leveraging the symplectic properties \eqref{eq:J_2n_properties}, \eqref{eq:sp_condition_derivation_2} becomes
%
\begin{equation*}
\Omega^{+}=-\Omega.
\end{equation*}
%
We notice that this definition of the tangent space at the identity is equivalent to the definition of the Lie algebra of $\mathrm{Sp}(2n)$ \cite[p.~3]{BendokatZimmermann2021}. It is given by 
%
\begin{equation}\label{eq:sp_Lie_algebra}
    \mathfrak{sp}(2n)\coloneqq T_{I}\mathrm{Sp}(2n)=\{\Omega\in \mathbb{R}^{2n\times2n} \;|\; \Omega^{+}=-\Omega\},
\end{equation}
%
where $\Omega$ is called the Hamiltonian matrix. 

Now, for a general $p \in \mathrm{Sp}(2n)$, $c(t)=pc_{I}(t)\in \mathrm{Sp}(2n)$, where $c(0)=p$. Therefore, by translation, we can define the tangent space of $\mathrm{Sp}(2n)$ at any point $p$  as
%
\begin{equation}\label{eq:sp_tangent_space}
\begin{split}
T_{p}\mathrm{Sp}(2n)&=\{\Omega p\in \mathbb{R}^{2n\times2n} \;|\; \Omega\in\mathfrak{sp}(2n)\},\\
&=\{\Omega p\in \mathbb{R}^{2n\times2n} \;|\; \Omega\in\mathfrak{sp}(2n)\}.
\end{split}
\end{equation}
%
We will verify that this makes sense by inserting $c\coloneqq c(t)$ in the condition of \eqref{eq:sp_def} to get
%
\begin{equation*}
c^{\mathrm{T}}p^{\mathrm{T}}Jpc=J.
\end{equation*}
%
We again take the derivative of to get
%
\begin{equation}\label{eq:sp_tangent_space_derivation_3}
\Omega ^{\mathrm{T}}p^{\mathrm{T}}Jp+p^{\mathrm{T}}Jp\Omega=0,
\end{equation}
where we used that $c(0)=0$, and the definition of $\Omega$. Now, multiplying \eqref{eq:sp_tangent_space_derivation_3} by $J^{\mathrm{T}}p$ from the left, and simplifying we again end up with $\Omega^{+}=-\Omega$, verifying our claim.

Now that we have defined the symplectic group, and its components that we will use throughout this report, we will move on to defining the symplectic Stiefel manifold.












% The Lie algebra of $\mathrm{Sp}(2n)$ is the symplectic groups' tangent space at the identity. It is given by 
% \begin{equation}\label{eq:sp_Lie_algebra}
%     \mathfrak{sp}(2n)\coloneqq \{\Omega\in \mathbb{R}^{2n\times2n} \;|\; \Omega^{+}=-\Omega\},
% \end{equation}
% where $\Omega$ is called the Hamiltonian matrix. 
% Now we can define the tangent space of $\mathrm{Sp}(2n)$ at any point $p$ \todo{maybe explain a little more how we get this tangent space} as
% \begin{equation}\label{eq:sp_tangent_space}
%     T_{p}\mathrm{Sp}(2n)=\{\Omega p\in \mathbb{R}^{2n\times2n} \;|\; \Omega\in\mathfrak{sp}(2n)\}.
% \end{equation}



