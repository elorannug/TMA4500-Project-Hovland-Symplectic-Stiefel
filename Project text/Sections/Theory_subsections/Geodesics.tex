\subsection{Geodesics}
To be able to define what it means to travel a (local) path of minimal length on any manifold, we need to define geodesics (defined generally in Definition \ref{def:geodesic}). They are the generalization of straight lines in Euclidean space, and since most optimization algorithms travel along paths, they are essential in transferring Euclidean optimization algorithms to the Riemannian domain. In this section we will begin by defining geodesics on $\mathrm{Sp}(2n)$. Through the lens of quotient manifolds, and our right invariant framework, we will define geodesics on $\mathrm{SpSt}(2n, 2k)$ from the geodesics on $\mathrm{Sp}(2n)$.

Following \cite[Prop.~2.1]{BendokatZimmermann2021}, given $p\in \mathrm{Sp}(2n)$,  $X\in T_{p}\mathrm{Sp}(2n)$ and the right-invariant Riemannian metric \eqref{eq:sp_metric}, the respective geodesic $\gamma(t)$ is defined as
\begin{equation*}
    \gamma(t)\coloneqq \mathrm{exp}\big(t(Xp^{+}-(Xp^{+}) ^{\mathrm{T}})\big)\mathrm{exp}\big(t(Xp^{+})^{\mathrm{T}}\big)p,
\end{equation*}
where $\gamma(0)=p$, $\dot{\gamma}(0)=X$ and $^{+}$ is the symplectic inverse (as in \eqref{eq:symplectic_inverse}). Here, $\operatorname{\exp}$ denotes the matrix exponential. 

To be able define geodesics from the right invariant metric on $\mathrm{SpSt}(2n, 2k)$ we need the following result. According to \cite[Cor.~7.46]{ONeill1983}, if $g_{p}^\mathrm{SpSt}$ has a horizontal tangent vector at every point, it projects to a Riemannian geodesic on $\mathrm{SpSt}(2n, 2k)$. 

The last preliminary definition we need is the following. In \cite[Lemma 3.11]{BendokatZimmermann2021} Bendokat and Zimmermann proves that if we, for a point $p \in \mathrm{Sp}(2n)$, define a geodesic $\gamma(t)$ through a horizontal tangent vector $X\in \operatorname{Hor}_{p}\mathrm{Sp}(2n)$, then $\dot{\gamma}(t)\in \operatorname{Hor}_{\gamma(t)}\mathrm{Sp}(2n)$. We call such a geodesic a \textit{horizontal geodesic}.

Following \cite[Prop. 3.12]{BendokatZimmermann2021}, we can now define a geodesic on $\mathrm{SpSt}(2n, 2k)$ through a horizontal geodesic $\gamma(t)\subseteq\mathrm{Sp}(2n)$. Let $q\in \mathrm{SpSt}(2n, 2k)$, $Y\in T_{q}\mathrm{SpSt}(2n, 2k)$, and $p\in \pi^{-1}(q)\subset \mathrm{Sp}(2n)$. Then the geodesic from $p$ in direction $Y$ is
%
\begin{equation}\label{eq:geodesic_spst}
\varphi(t)=\pi(\gamma(t))=\exp\big(t(\overline{\Omega}(Y)-\overline{\Omega}(Y)^{\mathrm{T}})\big)\exp\big(t \overline{\Omega}(Y)^{\mathrm{T}}\big)p.
\end{equation}
%
Now that we have an explicit expression for the geodesics on $\mathrm{SpSt}(2n, 2k)$, we can use them to define the Riemannian Hessian. However, one last component we need before we can express the Hessian is its predecessor: the Riemannian gradient. Therefore, in the next section we will define the Riemannian gradient on $\mathrm{SpSt}(2n, 2k)$, before moving on to the Hessian.