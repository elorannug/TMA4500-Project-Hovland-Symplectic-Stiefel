\subsection{Basic definitions}
% TODO: Give intro to section. What will be covered?
This section is designed to be a reference work to ensure that the reader has the necessary background to understand the optimization algorithms we will be studying. 

The optimization algorithm we will be studying is defined on a \textit{Riemannian manifold}. This is because the algorithms we will use are designed to utilize first and second order information, and we need to define what these concepts mean on a manifold. 


% manifolds
% J Lee smooth manifolds definition

% ⚠️ This is too basic!

% \begin{definition}[Smooth manifold]
%     Given a topological space $\mathcal{M}$, it is a topological manifold of dimension $n$ if
%     \begin{enumerate}
%         \item $\mathcal{M}$ is a Hausdorff space: for every pair of $p,q\in \mathcal{M}$, we can always find to disjoint subsets of $\mathcal{M}$, $U$ and $V$, such that $p\in U$ and $q\in V$.
%         \item $\mathcal{M}$ is second-countable: the topology of $\mathcal{M}$ has a countable basis.
%         \item $\mathcal{M}$ is locally Euclidean of dimension n: for each $p$ we have an open subset $U \subseteq \mathcal{M}$ containing $p$, and an open subset $\hat{U}\in \mathbb{R}^{n}$ such that there exists a homeomorphism $\phi:U \xrightarrow{}\hat{U}$.
%     \end{enumerate}
%     If we in addition have a notion of smoothness, meaning that the notion of differentiability is well-defined, we call $\mathcal{M}$ a smooth manifold.
% \end{definition}

% R manifolds
\begin{definition}[Riemannian manifold]
    As defined in \cite[def~2.6,~p.~179]{Boothby1975}: a \textup{smooth manifold} $\mathcal{M}$, defined as in \cite[p.~13]{Lee2012:1},  is a \textup{Riemannian manifold} if we can define a field of symmetric, positive definite, bilinear forms $g$, called the \textit{Riemannian metric}. By field we mean that $g_p$ is defined on the tangent space $T_p\mathcal{M}$ at each point $p\in \mathcal{M}$, as defined in \cite[def~2.1,~p.~178]{Boothby1975}. We will assume that $g$ is smooth, meaning that it is od class $\mathcal{C}^\infty$.
\end{definition}

\begin{definition}[Quotient space]\label{def:quotient_space}
    
\end{definition}

\begin{definition}[General Linear group]\label{def:general_linear_group}
    The \textup{real General Linear group} is defined as the set of all invertible matrices in $\mathbb{R}^{n\times n}$, denoted by $\mathrm{GL}(n)$. \cite[Example~9.11]{Boumal2023}
\end{definition}

\begin{definition}[Orthogonal group]\label{def:orthogonal_group}
    The \textup{real Orthogonal group} is defined as the set of all orthogonal matrices in $\mathbb{R}^{n\times n}$, denoted by $\mathrm{O}(n)$. \cite[p.~3]{Edelman1998}
\end{definition}
    


% TODO: Forklar at J_2n blir forenklet til J hvis det er klart fra konteksten

% TODO: Is this a paper? ⚠️
For the rest of this paper we denote $\mathcal{M}$ as being a Riemannian manifold.

\todo[inline]{After the basic definitions, talk about how the rest of the theory is a highlighted summary through BZ and JZ. The goal is to look at the findings in JZ, however it relies heavely on theory derived in BZ. It will be mentioned of some parts are from other works, or if they are original work.}
% optimization on R manif


% TODO: "Symplectic identity" J_2n

% TODO: Symplectic inverse






