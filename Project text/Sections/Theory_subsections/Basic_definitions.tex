\subsection{Foundational definitions}
This section is designed to be a reference work to set notation, and to make the rest of the report more readable in the sense that we do not take detours to define basic properties. Note that even though the elements of the symplectic group and symplectic stiefel manifold are matrices, the word "point" will be used to refer to a specific matrix, as the specific matrix is a point on the matrix manifold.


% \begin{definition}[General Linear group]\label{def:general_linear_group}
%     The \textup{real General Linear group} is defined as the set of all invertible matrices in $\mathbb{R}^{n\times n}$, denoted by $\mathrm{GL}(n)$. \cite[Example~9.11]{Boumal2023}
% \end{definition}

\begin{definition}[Orthogonal group]\label{def:orthogonal_group}
    The \textup{real Orthogonal group} is defined as the set of all orthogonal matrices in $\mathbb{R}^{n\times n}$, denoted by $\mathrm{O}(n)$. \cite[p.~3]{Edelman1998}
\end{definition}

\begin{definition}[Quotient manifold]\label{def:quotient_manifold}
    % Absil, Mahony, Sepulchre p.27
    We define the definition of quotient manifold as in \cite[p.~27]{AbsilMahonySepulchre2008}. Let $\mathcal{M}$ be a manifold equipped with the operation $\sim$ called the \textup{equivalence relation}. The equivalence relation has the following properties:
    \begin{enumerate}
        \item (reflexive) $p\sim p$ for all $p\in \mathcal{M}$,
        \item (symmetric) $p \sim q$ if and only if $q\sim p$ for all $q,p\in \mathcal{M}$, and
        \item (transitive) given $p\sim q$ and $q\sim r$ this implies that $p\sim r$ for all $p,q,r\in \mathcal{M}$.
    \end{enumerate}
    Given the set $[p]\coloneqq \{q\in \mathcal{M}\colon q\sim p\}$ called the \textup{equivalence class} of all points equivalent to $p$, the set
    \begin{equation*}
        \mathcal{M}/\sim~ \coloneqq \{[p] \;|\; p\in \mathcal{M}\}
    \end{equation*}
    is called the \textup{quotient of }$\mathcal{M}$\textup{ by }$\sim$. It is the set of all equivalence classes of $\sim$ in $\mathcal{M}$. The mapping $\pi\colon \mathcal{M}\xrightarrow{}\mathcal{M}/\sim$ called the \textup{natural-} or \textup{canonical projection}, defined by $p\mapsto[p]$. 
\end{definition}

\begin{definition}[Tangent Space]\label{def:tangent_space}
    Following \cite[Def.~8.33]{Boumal2023}, for a point $p$ on a smooth manifold $\mathcal{M}$, denote the set of smooth curves \cite[Def.~8.5]{Boumal2023} passing through $p$ at $t=0$ as $C_{p}$. This means that $\alpha(0)=p$ for all $\alpha\in C_{p}$. For $\alpha, \beta\in C_{p}$ we say that they are equivalent if 
    \begin{equation*}
        (\phi\circ \alpha)'(0)=(\phi\circ \beta)'(0),
    \end{equation*}
    meaning their derivatives match in a \textup{coordinate chart} (defined as in \cite[p.~4]{Lee2012:1}) if their derivatives in the coordinate chart at zero are equal. Denote this equivalence relation as $\alpha\sim \beta$. It has analogous properties to the equivalence relation in Definition \ref{def:quotient_manifold}. The equivalence class is defined as $[\alpha]=\left\{\beta\in C_{p} \;\middle|\; \alpha\sim \beta\right\}$. Every equivalence class is called a \textup{tangent vector} to $\mathcal{M}$ at $p$. The \textup{tangent space} at $p$ is the quotient set
    \begin{equation*}
        T_{p}\mathcal{M}=C_{p}/\sim~=\left\{[\alpha] \;\middle|\; \alpha\in C_{p}\right\}.
    \end{equation*}  
    We denote the \textup{tangent bundle} as $T\mathcal{M}=\bigsqcup_{p\in \mathcal{M}}T_{p}\mathcal{M}$, where $\bigsqcup$ denotes the disjoint union.
\end{definition}

% R manifolds
\begin{definition}[Riemannian manifold]\label{def:riemannian_manifold}
    As defined in \cite[def~2.6,~p.~179]{Boothby1975}: a \textup{smooth manifold} $\mathcal{M}$, as defined in \cite[p.~13]{Lee2012:1},  is a \textup{Riemannian manifold} if we can define a field of symmetric, positive definite, bilinear forms $g(\cdot,\cdot)$, called the \textit{Riemannian metric}. By field we mean that $g_p$ is defined on the tangent space $T_p\mathcal{M}$ at each point $p\in \mathcal{M}$ \cite[def~2.1,~p.~178]{Boothby1975}. We will assume that $g$ is smooth, meaning that it is of class $\mathcal{C}^\infty$.
\end{definition}


\begin{definition}[Vector field on a Riemannian manifold]\label{def:vector_field_on_manifold}
    Following Appendix A of \cite{JensenZimmermann2024}, a smooth vector field $\mathcal{X}\colon \mathcal{M}\xrightarrow{}T \mathcal{M},~p\mapsto \mathcal{X}(p)\in T_{p}\mathcal{M}$ on a Riemannian manifold $\mathcal{M}$ can be expressed through local coordinates as 
    \begin{equation*}
        \mathcal{X}(p)=\sum\limits_{i=1}^{n}\alpha_{i}\partial_{i}\eqqcolon \boldsymbol{\alpha}^{\mathrm{T}}\boldsymbol{\partial},
    \end{equation*}
    where $\boldsymbol{\alpha}\in \mathbb{R}^{n}$, and $\boldsymbol{\partial}$ is the canonical basis of $T_{p}\mathcal{M}$.
\end{definition}

\begin{definition}[Whitney sum]\label{def:Whitney_sum}
    Restricting the definition in \cite[p.114]{Bourles2019} to our scope, for a manifold $\mathcal{M}$ define subsets of the tangent bundle (see Definition \ref{def:tangent_space}) $\mathcal{X},\mathcal{Y}\subseteq T\mathcal{M}$. The Whitney sum is defined as 
    \begin{equation*}
        \mathcal{X}\oplus \mathcal{Y}\coloneqq\bigsqcup_{p\in \mathcal{M}}\mathcal{X}_{p}\times \mathcal{Y}_{p},
    \end{equation*}
    where $\bigsqcup$ denotes the disjoint union. 
\end{definition}

\begin{definition}[Horizontal \& Vertical Space]\label{def:horizontal_vertical_space}
    Using Definition \ref{def:quotient_manifold}, given a Riemannian manifold $\overline{\mathcal{M}}$ with Riemannian metric $\overline{g}$, denote a quotient manifold of $\overline{\mathcal{M}}$ as $\mathcal{M}=\overline{\mathcal{M}}/\sim$. Following the definitions in Absil et al. \cite[p.~43]{AbsilMahonySepulchre2008}, for a point $p\in \mathcal{M}$, the equivalence class $[p]=\pi^{-1}(p)$ induces an embedded submanifold of $\overline{\mathcal{M}}$ (see Definition \ref{def:quotient_manifold}), hence it admits a tangent space,
    \begin{equation*}
        \mathcal{V}_{\overline{p}}=T_{\overline{p}}(\pi^{-1}(p))=\operatorname{ker}(\operatorname{D}\pi(p))
    \end{equation*}
    \cite[p.~4]{JensenZimmermann2024} named the \textup{vertical space} at $\overline{p}$. Canonically chosen as the orthogonal complement of $\mathcal{V}_{\overline{p}}$ in $T_{\overline{p}}\overline{\mathcal{M}}$, the \textup{horizontal space} \cite[p.~48]{AbsilMahonySepulchre2008} is defined as %  p.48
    \begin{equation*}
        \mathcal{H}_{\overline{p}}\coloneqq \mathcal{V}_{\overline{p}}^\perp=\{Y_{\overline{p}}\in T_{\overline{p}}\overline{\mathcal{M}} \;|\; \overline{g}(Y_{\overline{p}},Z_{\overline{p}})=0\quad\forall\quad Z_{\overline{p}}\in \mathcal{V}_{\overline{p}} \}.
    \end{equation*}
    The \textup{horizontal lift} at $\overline{p}\in\pi^{-1}(p)$ of a tangent vector $X_{p}\in T_{p}\mathcal{M}$ is the unique tangent vector $X_{\overline{p}}\in \mathcal{H}_{\overline{p}}$ that satisfies $\mathrm{D}\pi(\overline{p})[{X}_{\overline{p}}]=X_p$. Note that given the horizontal space on $\overline{\mathcal{M}}$, $\mathcal{H}_{\overline{p}}\oplus \mathcal{V}_{\overline{p}}=T_{\overline{p}}\mathcal{M}$, where $\oplus$ denotes the Whitney sum as in Definition \ref{def:Whitney_sum}. 
\end{definition}

\begin{definition}[Riemannian connection]\label{def:riemannian_connection}
    The \textup{Riemanian connection}, also known as the \textup{Levi-Civita connection}, is the unique affine connection which is torsion free, and metric compatible \cite[Def.~6.4]{Tu2017}. In Appendix A of \cite{JensenZimmermann2024}, denoting $\mathfrak{X}(\mathcal{M})$ as the space of smooth vector fields on $\mathcal{M}$, it is defined as the unique $\mathbb{R}$-bilinear smooth map on $\mathcal{M}$ with riemannian metric $\langle \cdot, \cdot\rangle_{p}$
    \begin{equation*}
        \nabla\colon \mathfrak{X}(\mathcal{M})\times \mathfrak{X}(\mathcal{M})\xrightarrow{}\mathfrak{X}(\mathcal{M}),\quad (X,Y)\mapsto \nabla_{X}Y,
    \end{equation*}
    such that the following properties hold. Given $X,Y,Z\in \mathfrak{X}(\mathcal{M})$, and $f\in \mathcal{C}^{\infty}(M)$, $\nabla_{X}Y$has the following properties:
    \begin{enumerate}
        \item (first argument linearity) $\nabla_{fX}Y=f \nabla_{X}Y$, 
        \item (Leibnitz) $\nabla_{X}(fY)=(Xf)Y+f \nabla_{X}Y$, 
        \item (torsion free) $\nabla_{X}Y-\nabla_{Y}X=[X,Y]$, where $[\cdot,\cdot]$ is the Lie bracket \cite[p.~186]{Lee2012:1}, and
        \item (metric compatibility) $Z \langle X,Y\rangle=\langle \nabla_{Z}X,Y\rangle+\langle X,\nabla_{Z}Y\rangle$.
    \end{enumerate}
\end{definition}

\begin{definition}[Geodesics]\label{def:geodesic}
    By \cite[Def.~14.1]{Tu2017} assume we have a Riemannian manifold $\mathcal{M}$ equipped with an affine connection $\nabla$ (as in Definition \ref{def:riemannian_connection}). A curve $c\colon I\to \mathcal{M}$, where $I$ is an open, closed or half-open interval of $\mathbb{R}$, is a \textup{geodesic} if the covariant derivative (defined in Section \ref{sec:riemannian_hessian}) of $c'(t)$ is zero for all $t$. 
\end{definition}

\begin{definition}[Riemannian gradient]\label{def:riemannian_gradient}
    As defined in \cite[Def.~3.58]{Boumal2023}, for the Riemannian manifold $\mathcal{M}$, let $f\colon\mathcal{M}\to \mathbb{R}$ be a smooth function. The \textup{Riemannian gradient} of $f$ is the vector field $\operatorname{grad}f$ on $\mathcal{M}$ defined uniquely by for all $(p,X)\in T\mathcal{M}$,
    %
    \begin{equation*}
    \operatorname{D}f(p)[X]=\langle X,\operatorname{grad}f(p) \rangle _{p},
    \end{equation*}
    %
    where $\operatorname{D}$ denotes the differential of $f$ at $p$ meaning $(f\circ c)'(0)$ for som curve $c(t)$ like in Definition \ref{def:tangent_space} \cite[Def.~3.34]{Boumal2023}. The $\langle \cdot,\cdot \rangle_{p}$ denotes the Riemannian metric at the point $p$. 
\end{definition}

\begin{definition}[Christoffel symbols]\label{def:christoffel_symbols}
    The method we will employ to completely describe a connection (as defined in Definition \ref{def:riemannian_connection}) locally is to describe them through $\textup{Christoffel symbols}$. Following the definition in \cite[p.~100]{Tu2017}, let $\nabla$ be an affine connection on $\mathcal{M}$. Denote a coordinate vector field on the coordinate open set $(U,p^{1},\dots,p^{n})\subseteq\mathcal{M}$ by $\partial_{i}\coloneqq \partial/\partial p^i$ as in \cite[Appendix~A.3]{Tu2017}. In this coordinate frame there exist the numbers called Christoffel symbols, $\Gamma_{ij}^{k}$, defined through the following
    \begin{equation*}
        \nabla_{\partial_{i}}\partial_{j}=\sum\limits_{k=1}^{n}\Gamma_{ij}^{k}\partial_{k}\eqqcolon\mathbf{\Gamma}_{ij}^{\mathrm{T}}\mathbf{\partial}.    
    \end{equation*}
\end{definition}

\begin{definition}[Retraction]\label{def:retraction}
    Following \cite[Def.~3.47]{Boumal2023}, a retraction on a smooth manifold $\mathcal{M}$ is a smooth map,
    $$\mathcal{R}\colon T \mathcal{M}\xrightarrow{}\mathcal{M},\quad(p,X)\mapsto \mathcal{R}_{p}(X)$$
    such that every curve generated from $c(t)=\mathcal{R}_{p}(tX)$ satisfies $c(0)=p$ and $\dot{c}(0)=X$. Equivalently the conditions can be stated as in \cite[p.~40]{Boumal2023} without the use of curves. For all $p\in \mathcal{M}$, $\mathcal{R}_{p}(0)=p$, and $\operatorname{D}\mathcal{R}_{p}(0)\colon T_{p}\mathcal{M}\xrightarrow{}T_{p}\mathcal{M}$, $\operatorname{D}\mathcal{R}_{p}(0)[X]=X$ is the identify map. 
\end{definition}

Now that we have defined som foundational definitions to this report, we will proceed with defining the symplectic group and the symplectic Stiefel manifold, the objects of interest in this report. Note that for the rest of this project report we denote $\mathcal{M}$ as being a Riemannian manifold unless stated otherwise.








