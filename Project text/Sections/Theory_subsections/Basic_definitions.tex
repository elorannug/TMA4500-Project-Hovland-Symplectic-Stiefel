\subsection{Basic definitions}
% TODO: Give intro to section. What will be covered?
This section is designed to be a reference work to ensure that the reader has the necessary background to understand the optimization algorithms we will be studying. 

The optimization algorithm we will be studying is defined on a \textit{Riemannian manifold}. This is because the algorithms we will use are designed to utilize first and second order information, and we need to define what these concepts mean on a manifold. 


% manifolds
% J Lee smooth manifolds definition

% ⚠️ This is too basic!

% \begin{definition}[Smooth manifold]
%     Given a topological space $\mathcal{M}$, it is a topological manifold of dimension $n$ if
%     \begin{enumerate}
%         \item $\mathcal{M}$ is a Hausdorff space: for every pair of $p,q\in \mathcal{M}$, we can always find to disjoint subsets of $\mathcal{M}$, $U$ and $V$, such that $p\in U$ and $q\in V$.
%         \item $\mathcal{M}$ is second-countable: the topology of $\mathcal{M}$ has a countable basis.
%         \item $\mathcal{M}$ is locally Euclidean of dimension n: for each $p$ we have an open subset $U \subseteq \mathcal{M}$ containing $p$, and an open subset $\hat{U}\in \mathbb{R}^{n}$ such that there exists a homeomorphism $\phi:U \xrightarrow{}\hat{U}$.
%     \end{enumerate}
%     If we in addition have a notion of smoothness, meaning that the notion of differentiability is well-defined, we call $\mathcal{M}$ a smooth manifold.
% \end{definition}

\begin{definition}[General Linear group]\label{def:general_linear_group}
    The \textup{real General Linear group} is defined as the set of all invertible matrices in $\mathbb{R}^{n\times n}$, denoted by $\mathrm{GL}(n)$. \cite[Example~9.11]{Boumal2023}
\end{definition}

\begin{definition}[Orthogonal group]\label{def:orthogonal_group}
    The \textup{real Orthogonal group} is defined as the set of all orthogonal matrices in $\mathbb{R}^{n\times n}$, denoted by $\mathrm{O}(n)$. \cite[p.~3]{Edelman1998}
\end{definition}

\begin{definition}[Tangent Space]\label{def:tangent_space}
    
\end{definition}

% R manifolds
\begin{definition}[Riemannian manifold]\label{def:riemannian_manifold}
    As defined in \cite[def~2.6,~p.~179]{Boothby1975}: a \textup{smooth manifold} $\mathcal{M}$, as defined in \cite[p.~13]{Lee2012:1},  is a \textup{Riemannian manifold} if we can define a field of symmetric, positive definite, bilinear forms $g(\cdot,\cdot)$, called the \textit{Riemannian metric}. By field we mean that $g_p$ is defined on the tangent space $T_p\mathcal{M}$ at each point $p\in \mathcal{M}$ \cite[def~2.1,~p.~178]{Boothby1975}. We will assume that $g$ is smooth, meaning that it is of class $\mathcal{C}^\infty$.
\end{definition}


\begin{definition}[Vector field on Riemannian manifold]\label{def:vector_field_on_manifold}
    Following Appendix A of \cite{JensenZimmermann2024}, a smooth vector field $\mathcal{X}:\mathcal{M}\xrightarrow{}T \mathcal{M},~p\mapsto \mathcal{X}(p)\in T_{p}\mathcal{M}$ on a Riemannian manifold $\mathcal{M}$ can be expressed through local coordinates as 
    \begin{equation*}
        \mathcal{X}(p)=\sum\limits_{i=1}^{n}\alpha_{i}\partial_{i}\eqqcolon \mathbf{\alpha}^{\mathrm{T}}\mathbf{\partial},
    \end{equation*}
    where $\mathbf{\alpha}\in \mathbb{R}^{n}$, and $\mathbf{\partial}$ is the canonical basis of $T_{p}\mathcal{M}$.
    
\end{definition}

\begin{definition}[Quotient manifold]\label{def:quotient_manifold}
    % Absil, Mahony, Sepulchre p.27
    We define the definition of quotient manifold as in \cite[p.~27]{AbsilMahonySepulchre2008} Let $\mathcal{M}$ be a manifold equipped with the operation $\sim$ called the \textup{equivalence relation}. The equivalence relation has the following properties:
    \begin{enumerate}
        \item (reflexive) $p\sim p$ for all $p\in \mathcal{M}$,
        \item (symmetric) $p \sim q$ if and only if $q\sim p$ for all $q,p\in \mathcal{M}$, and
        \item (transitive) given $p\sim q$ and $q\sim r$ this implies that $p\sim r$ for all $p,q,r\in \mathcal{M}$.
    \end{enumerate}
    Given the set $[p]\coloneqq \{q\in \mathcal{M}:q\sim p\}$ called the \textup{equivalence class} of all points equivalent to $p$, the set
    \begin{equation*}
        \mathcal{M}/\sim \coloneqq \{[p] \;|\; p\in \mathcal{M}\}
    \end{equation*}
    is called the \textup{quotient of }$\mathcal{M}$\textup{ by }$\sim$. It is the set of all equivalence classes of $\sim$ in $\mathcal{M}$. The mapping $\pi:\mathcal{M}\xrightarrow{}\mathcal{M}/\sim$ called the \textup{natural-} or \textup{canonical projection}, defined by $p\mapsto[p]$ . 
\end{definition}

\begin{definition}[Horizontal \& Vertical Space]\label{def:horizontal_vertical_space}
    Given a Riemannian manifold $\overline{\mathcal{M}}$ with Riemannian metric $\overline{g}$, denote a quotient manifold of $\overline{\mathcal{M}}$ as $\mathcal{M}=\overline{\mathcal{M}}/\sim$. 
    %page 43 for V space
    Following the definitions in Absil et al. \cite[p.~43]{AbsilMahonySepulchre2008}, for a point $p\in \mathcal{M}$, the equivalence class $[p]=\pi^{-1}(p)$ induces an embedded submanifold of $\overline{\mathcal{M}}$ (see \Cref{def:quotient_manifold}), hence it admits a tangent space,
    \begin{equation*}
        \mathcal{V}_{\overline{p}}=T_{\overline{p}}(\pi^{-1}(p))
    \end{equation*}
    named the \textup{vertical space} at $\overline{p}$. Canonically chosen as the orthogonal complement of $\mathcal{V}_{\overline{p}}$ in $T_{\overline{p}}\overline{\mathcal{M}}$, the \textup{horizontal space} \cite[p.~48]{AbsilMahonySepulchre2008} is defined as %  p.48
    \begin{equation*}
        \mathcal{H}_{\overline{p}}\coloneqq \mathcal{V}_{\overline{p}}^\perp=\{Y_{\overline{p}}\in T_{\overline{p}}\overline{\mathcal{M}} \;|\; \overline{g}(Y_{\overline{p}},Z_{\overline{p}})=0\quad\forall\quad Z_{\overline{p}}\in \mathcal{V}_{\overline{p}} \}.
    \end{equation*}
    The \textup{horizontal lift} at $\overline{p}\in\pi^{-1}(p)$ of a tangent vector $X_{p}\in T_{p}\mathcal{M}$ is the unique tangent vector $X_{\overline{p}}\in \mathcal{H}_{\overline{p}}$ that satisfies $\mathrm{D}\pi(\overline{p})[{X}_{\overline{p}}]=X_p$. Note that given the horizontal space on $\overline{\mathcal{M}}$, $\mathcal{H}_{\overline{p}}\oplus \mathcal{V}_{\overline{p}}=T_{\overline{p}}\mathcal{M}$, where $\oplus$ denotes the \todo{define w sum! i wrote it in obsidian in the H and V space file!}Whitney sum. 
\end{definition}

\begin{definition}[Riemannian connection]\label{def:riemannian_connection}
    The \textup{Riemanian connection}, also known as the \textup{Levi-Civita connection}, is the unique affine connection which is torsion free, and metric compatible \cite[Def.~6.4]{Tu2017}. In Appendix A of \cite{JensenZimmermann2024}, denoting $\mathfrak{X}(\mathcal{M})$ as the space of smooth vector fields on $\mathcal{M}$, it is defined as the unique $\mathbb{R}$-bilinear smooth map on $\mathcal{M}$ with riemannian metric $\langle \cdot, \cdot\rangle_{p}$
    \begin{equation*}
        \nabla:\mathfrak{X}(\mathcal{M})\times \mathfrak{X}(\mathcal{M})\xrightarrow{}\mathfrak{X}(\mathcal{M}),\quad (X,Y)\mapsto \nabla_{X}Y,
    \end{equation*}
    such that the following properties hold. Given $X,Y,Z\in \mathfrak{X}(\mathcal{M})$, and $f\in \mathcal{C}^{\infty}(M)$, $\nabla_{X}Y$has the following properties:
    \begin{enumerate}
        \item (first argument linearity) $\nabla_{fX}Y=f \nabla_{X}Y$, 
        \item \todo{In JZ this was stated wrongly}(Leibnitz) $\nabla_{X}(fY)=(Xf)Y+f \nabla_{X}Y$, 
        \item (torsion free) $\nabla_{X}Y-\nabla_{Y}X=[X,Y]$, where $[\cdot,\cdot]$ is the \todo{citation}Lie bracket, and
        \item (metric compatibility) $Z \langle X,Y\rangle=\langle \nabla_{Z}X,Y\rangle+\langle X,\nabla_{Z}Y\rangle$.
    \end{enumerate}
\end{definition}

\begin{definition}[Christoffel symbols]\label{def:christoffel_symbols}
    The method we will employ to completely describe a connection (as defined in \Cref{def:riemannian_connection}) locally is to describe them through $\textup{Christoffel symbols}$. Following the definition of \cite[p.~100]{Tu2017}, let $\nabla$ be an affine connection on $\mathcal{M}$. Denote a \todo{do i need a source to define this?}coordinate vector field on the coordinate open set $(U,p^{1},\dots,p^{n})\subseteq\mathcal{M}$ by $\partial_{i}\coloneqq \partial/\partial p^i$. In this coordinate frame there exist the numbers called Christoffel symbols, $\Gamma_{ij}^{k}$, defined through the following
    \begin{equation*}
        \nabla_{\partial_{i}}\partial_{j}=\sum\limits_{k=1}^{n}\Gamma_{ij}^{k}\partial_{k}\eqqcolon\mathbf{\Gamma}_{ij}^{\mathrm{T}}\mathbf{\partial}.    
    \end{equation*}
\end{definition}

\begin{definition}[Retraction]\label{def:retraction}
    Following \cite[Def.~3.47]{Boumal2023}, a retraction on a smooth manifold $\mathcal{M}$ is a smooth map,
    $$\mathcal{R}:T \mathcal{M}\xrightarrow{}\mathcal{M},\quad(p,X)\mapsto \mathcal{R}_{p}(X)$$
    such that every curve generated from $c(t)=\mathcal{R}_{p}(tX)$ satisfies $c(0)=p$ and $\dot{c}(0)=X$. Equivalently the conditions can be stated as in \cite[p.~40]{Boumal2023} without the use of curves. For all $p\in \mathcal{M}$, $\mathcal{R}_{p}(0)=p$, and $\operatorname{D}\mathcal{R}_{p}(0):T_{p}\mathcal{M}\xrightarrow{}T_{p}\mathcal{M}$, $\operatorname{D}\mathcal{R}_{p}(0)[X]=X$ is the identify map. 
\end{definition}

\begin{definition}[]\label{def:}
    
\end{definition}

\begin{definition}[]\label{def:}
    
\end{definition}
    


% TODO: Forklar at J_2n blir forenklet til J hvis det er klart fra konteksten

% TODO: Is this a paper? ⚠️
For the rest of this paper we denote $\mathcal{M}$ as being a Riemannian manifold.

\todo[inline]{After the basic definitions, talk about how the rest of the theory is a highlighted summary through BZ and JZ. The goal is to look at the findings in JZ, however it relies heavely on theory derived in BZ. It will be mentioned of some parts are from other works, or if they are original work.}
% optimization on R manif


% TODO: "Symplectic identity" J_2n

% TODO: Symplectic inverse






