\subsection{Basic definitions}
% TODO: Give intro to section. What will be covered?
This section is designed to be a reference work to ensure that the reader has the necessary background to understand the optimization algorithms we will be studying. 

The optimization algorithm we will be studying is defined on a \textit{Riemannian manifold}. This is because the algorithms we will use are designed to utilize first and second order information, and we need to define what these concepts mean on a manifold. 


% manifolds
% J Lee smooth manifolds definition

% ⚠️ This is too basic!

% \begin{definition}[Smooth manifold]
%     Given a topological space $\mathcal{M}$, it is a topological manifold of dimension $n$ if
%     \begin{enumerate}
%         \item $\mathcal{M}$ is a Hausdorff space: for every pair of $p,q\in \mathcal{M}$, we can always find to disjoint subsets of $\mathcal{M}$, $U$ and $V$, such that $p\in U$ and $q\in V$.
%         \item $\mathcal{M}$ is second-countable: the topology of $\mathcal{M}$ has a countable basis.
%         \item $\mathcal{M}$ is locally Euclidean of dimension n: for each $p$ we have an open subset $U \subseteq \mathcal{M}$ containing $p$, and an open subset $\hat{U}\in \mathbb{R}^{n}$ such that there exists a homeomorphism $\phi:U \xrightarrow{}\hat{U}$.
%     \end{enumerate}
%     If we in addition have a notion of smoothness, meaning that the notion of differentiability is well-defined, we call $\mathcal{M}$ a smooth manifold.
% \end{definition}

% R manifolds
\subsubsection{Riemannian manifolds}
% Definition from WM Boothby page 191
As defined in [\cite{Boothby}] A smooth manifold (see [\cite{Lee2012}] smooth for definition) $\mathcal{M}$ is a \textit{Riemannian manifold} if we can define a field of symmetric, positive definite, bilinear forms $g$, called the \textit{Riemannian metric}. By field we mean that $g_p$ is defined on the tangent space $T_p\mathcal{M}$ at each point $p\in \mathcal{M}$.


% TODO: Forklar at J_2n blir forenklet til J hvis det er klart fra konteksten

% TODO: Is this a paper? ⚠️
For the rest of this paper we denote $\mathcal{M}$ as being a Riemannian manifold.
% optimization on R manif


% TODO: "Symplectic identity" J_2n

% TODO: Symplectic inverse






