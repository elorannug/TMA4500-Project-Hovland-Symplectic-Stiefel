\subsection{Riemannian Hessian of the Symplectic Stiefel manifold}
\todo[inline]{Define the remaining preliminaries for Hessian. Make an intro to section. loosly following Appendix A in Jz}
If one knows more about the manifold, one can make more assumptions for an optimization algorithm. The hope would be that these additional assumptions would make the algorithm more efficient. In the Euclidean setting, the Hessian gives us curvature information about the cost function. Generalizing the Hessian to the Riemannian setting it gives us something similar, and we can generalize many popular algorithms. In this section we will therefore define the Riemannian Hessian of the symplectic manifold. We begin by defining what it means to take a second derivative on a manifold. After defining the necessary tools, we will give an analytical expression for the Riemannian Hessian. 

For two smooth vector fields, $\mathcal{X}(p)=\boldsymbol{\alpha}^{\mathrm{T}}\boldsymbol{\partial}$ and $\mathcal{Y}(p)=\boldsymbol{\beta}^{\mathrm{T}}\boldsymbol{\partial}$ defined as in Definition \ref{def:vector_field_on_manifold}, the covariant derivative (defined through the Riemannian connection by Definition \ref{def:riemannian_connection}) written in local coordinates is
\begin{equation*}
    \nabla_{\mathcal{X}}\mathcal{Y}=\sum\limits_{i=1}^{n}\sum\limits_{j=1}^{n}\alpha_{i}\partial_{i}(\beta_{j})\partial_{j}+\alpha_{i}\beta_{j}\sum\limits_{k=1}^{n}\Gamma_{ij}^{k}\partial_{k}.
\end{equation*}
In preparation for the Hessian, we include the description of the covariant derivative from \cite[p.~96]{Tu2017} to restrict the covariant derivative further. We want to define the covariant derivative of a vector field along a curve $c(t)$. $c(t)$ is the smooth curve, $c\colon I\xrightarrow{}\mathcal{M},$ $t\mapsto(\gamma_{1}(t), \dots, \gamma_{n}(t))$, where $I \coloneqq [a,b]\subseteq \mathbb{R}$. Since we have an affine connection on $\mathcal{M}$, the following unique map exists:
\begin{equation*}
    \frac{D}{\mathrm{d}t}\colon \Gamma(T \mathcal{M}|_{c(t)})\xrightarrow{}\Gamma(T \mathcal{M}|_{c(t)}),
\end{equation*}
where $\Gamma(T \mathcal{M}|_{c(t)})$ denotes the the vector space of all smooth vector fields along $c(t)$. If $V\in \Gamma(T \mathcal{M}|_{c(t)})$ is induced by $\mathcal{X}$, meaning $V(t)=\mathcal{X}|_{c(t)}$, then
\begin{equation*}
    \frac{DV}{\mathrm{d}t}(t)=\nabla_{\dot{c}(t)}\mathcal{X}=\dot{\alpha}(t)+\Gamma(\alpha(t), \dot{\gamma}(t)),\quad \Gamma(u,v)=\begin{bmatrix}u ^{\mathrm{T}}\Gamma^{1}v \\ \vdots \\ u ^{\mathrm{T}}\Gamma^{n}v\end{bmatrix}.
\end{equation*}
$\Gamma(u,v)$ is called the $\textit{Christoffel function}$. \todo{i felt JZ was unclear. Is this correct?}If $\dot{c}(t)$ is a \todo{refs}geodesic, the expression above reduces to 
\begin{equation}\label{eq:christoffel_symbols_through_geodesic}
    \ddot{\gamma}(t)=-\Gamma(\dot{\gamma}(t),\dot{\gamma}(t)),
\end{equation}
since by definition geodesics must satisfy $\tfrac{D}{\mathrm{D}t}\dot{\gamma}(t)=0$ and $\dot{\alpha}(t)=\dot{\gamma}(t)$. Importantly, once we have found the Christoffel symbols through \eqref{eq:christoffel_symbols_through_geodesic}, we can still use them for curves that are not geodesics. This is because the Christoffel symbols only depend on the Riemannian metric, and the local coordinates. To do this, we recover the Christoffel function for two different inputs through polarization \cite[p.~312]{Edelman1998} 
\begin{equation*}
    \Gamma(X,Y)=\frac{1}{4}\big(\Gamma(X+Y,X+Y)-\Gamma(X-Y,X-Y)\big),
\end{equation*}
where $X,Y\in \Gamma(T \mathcal{M}|_{c(t)})$. 
\todo[inline]{i do not understand polarization.}


To find the Christoffel symbols for $\mathrm{SpSt}(2n, 2k)$ with respect to the right invariant metric $g$ defined in \eqref{eq:spst_metric}, we differentiate the geodesic formula from \todo{ref geod.}ref, and use \eqref{eq:christoffel_symbols_through_geodesic} to achieve the following formula,
\begin{equation*}
    \Gamma(X,X)=-\ddot{\gamma}(0)=-(\overline{\Omega}(X)-\overline{\Omega}(X)^{\mathrm{T}})(X+\overline{\Omega}(X)^{\mathrm{T}}p)-(\overline{\Omega}(X)^{\mathrm{T}})^{2}p.
\end{equation*}
Here $X=\dot{\gamma}(0)\in T_{p}\mathrm{SpSt}(2n, 2k)$, $p\in \mathrm{SpSt}(2n, 2k)$, and $\overline{\Omega}(X)$ is as in \todo{define this somewhere}ref. 
With our metric $g$, the Hessian at $p$ of a smooth function $f\colon \mathrm{SpSt}(2n, 2k)\xrightarrow{}\mathbb{R}$ is the endomorphism 
\begin{gather*}
    \operatorname{Hess}f(p)\colon T_{p}\mathrm{SpSt}(2n, 2k)\xrightarrow{}T_{p}\mathrm{SpSt}(2n, 2k),  \\
    \operatorname{Hess}f(p)[X]=\left.\frac{\mathrm{d}}{\mathrm{d}t}\operatorname{grad}f(c(t))\right|_{t=0}+\Gamma(\operatorname{grad}f(p),X),
\end{gather*}
where $\operatorname{grad}f(\cdot)$ is as in \ref{eq:rie_grad}. $c(t)\in \mathrm{SpSt}(2n, 2k)$ is an arbitrary curve such that $c(0)=p$ and $c'(0)=X$ % Curve definition should be somewhere else