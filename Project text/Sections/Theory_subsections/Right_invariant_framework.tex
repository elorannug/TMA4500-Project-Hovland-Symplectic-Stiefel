\subsection{Right-invariant framework}
% BZ: This metric is invariant under the group action of Sp(2k, R)
% from the right and induces therefore a Riemannian metric on the symplectic subspaces
% that will be considered in Subsection 4.2.

% Right-invariant metric
\todo[inline]{many of the usual SpSt metrics do not have geodesics, therefore... We need a proper metric for $\mathrm{SpSt}(2n, 2k)$ in the sense that we need a metric that allows us to perform optimization on this manifold. To this end we need a metric that makes it possible to derive geodesics, as opposed to (BZ sources). The following metric defined in BZ fulfils our criteria.}

The goal for this section is to use the right-invariant metric defined on the symplectic group to define an appropriate metric on the symplectic Stiefel manifold. To achieve this we will use a \textit{horizontal lift} to define a metric on $\mathrm{SpSt}(2n, 2k)$ through \ref{eq:sp_metric}. Split $T_{p}\mathrm{Sp}(2n,\mathbb{R})$ into to parts: the horizontal- and vertical part, with respect to $g^\mathrm{Sp}_{p}$ and $\pi$: % TODO: Define \pi! BZ (3.2)

\begin{equation}\label{eq:spst_split}
    T_{p}\mathrm{Sp}(2n)=\operatorname{Ver}^{\pi}_{p}\mathrm{Sp}(2n)\oplus \operatorname{Hor}^{\pi}_{p}\mathrm{Sp}(2n).
\end{equation}
\todo[inline]{Define $\operatorname{Ver}^{\pi}_{p}\mathrm{Sp}(2n)$ and $\operatorname{Hor}^{\pi}_{p}\mathrm{Sp}(2n)$ in a smart way. Maybe just a reference if I do not have the space.}

The point-wise right-invariant Riemannian metric on $\mathrm{SpSt}(2n, 2k)$ is defined as the mapping $g_{p}:T_{p}\mathrm{SpSt}(2n, 2k)\times T_{p}\mathrm{SpSt}(2n, 2k)\xrightarrow{}\mathbb{R}$, $g_{p}(X_{1},X_{2})\coloneqq g^{\mathrm{Sp}}_{p}((X_{1})^{\mathrm{hor}}_{p},(X_{2})^{\mathrm{hor}}_{p})$. More explicitly

\begin{equation}\label{eq:spst_metric}
    g_{p}(X_{1},X_{2})=\operatorname{tr}\left(X_{1}^{T}\left(I_{2n}- \frac{1}{2}J_{2n}^{T}p(p^{T}p)^{-1}p^{T}J_{2n}\right)X_{2}(p^{T}p)^{-1}\right),
\end{equation}
for $X_{1},X_{2}\in T_{p}\mathrm{SpSt}(2n, 2k)$. For this metric, $\pi$ denotes a Riemannian submersion.








\begin{align}
    &p^{\mathrm{T}}J\nabla f(p)p^{\mathrm{T}}p+p^{\mathrm{T}}JJp(\nabla f(p))^{\mathrm{T}}Jp+p^{\mathrm{T}}p(\nabla f(p))^{\mathrm{T}}Jp+p^{\mathrm{T}}J^{\mathrm{T}}\nabla f(p)p^{\mathrm{T}}J^{\mathrm{T}}Jp\\
    &= -p^{\mathrm{T}}J^{\mathrm{T}}\nabla f(p)p^{\mathrm{T}}J^{\mathrm{T}}Jp
\end{align}

%%% Riemannian Hessian TODO: På Obsidian