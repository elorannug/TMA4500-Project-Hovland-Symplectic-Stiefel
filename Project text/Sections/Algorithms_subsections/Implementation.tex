\subsection{Implementation}
In this section we will outline how we will conduct our experiments. We will first define which retractions and algorithms we will test, as well as presenting the main packages we will use to implement the experiments. Finally we will cite where one can find the code used in the following experiments. 

Before comparing algorithms, we will first look at the feasibility off the different retractions defined in Section \ref{sec:theory_to_application}. These are the Cayley retraction, the pseudo-Riemannian geodesic, and the Riemannian geodesic. The motivation for this experiment is to test how the different retractions perform in terms of speed and accuracy, with the goal of finding which retraction to use in the subsequent experiments.

We will use three algorithms for our experiments. The first is the gradient descent algorithm using the Armijo condition to find appropriate step sizes, denoted as GD. The second is the trust-region algorithm using the exact Riemannian Hessian $\operatorname{Hess}f(p_{k})[X]$ as in \eqref{eq:riemannian_hessian}, denoted as TR-1. The final algorithm, which we will denote as TR-2, will differ from TR-1 by using the approximate Hessian $\operatorname{Proj}_{p_{k}}(\operatorname{D}\overline{\operatorname{grad}}f(p_{k})[X])$ defined in \eqref{eq:approximate_hessian}. The shorthand TR will be used when referring to both TR-1 and TR-2.

The code for the experiments are mainly using the two packages Manifolds.jl \cite{AxenBaranBergmannRzecki:2023} and Manopt.jl \cite{Bergmann2022}. Manifolds.jl provides us with a useful framework to be able to work with their implementation of the symplectic Stiefel manifold. Manopt.jl provides us with a high level framework to be able to implement the optimization algorithms. An effort has been made to ensure the parameters in functions and objects defined in the utilized libraries are consistent as to align with the theory presented in this project report. 

To measure the performance of the algorithms in terms of computational time, the function \text{@benchmark} from the package BenchmarkTools.jl \cite{BenchmarkTools:2016} will be used.

Although many of the necessary tools for the experiments are already implemented in Manifolds.jl and Manopt.jl, the following components were implemented manually from the theory presented in this project report. These are the pseudo-Riemannian geodesic, the Riemannian Hessian and its components, and the approximate Hessian. The following is already implemented in the two libraries, but were manually implemented for debugging purposes: Cayley retraction, and the projection of Euclidean gradient to the Riemannian gradient. 

All files used to produce the data in this report are available in the following \href{https://github.com/kellertuer/TMA4500-Project-Hovland-Symplectic-Stiefel.git}{github repository} \cite{Hovland2024}. 