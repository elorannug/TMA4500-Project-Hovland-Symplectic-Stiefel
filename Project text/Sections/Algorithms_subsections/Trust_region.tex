\subsection{Riemannian trust-region method}
Since the gradient descent method only uses first order information, a natural question would be to investigate how a method that utilizes second order information would perform. The criterion for this second order method to be considered better, for a specific problem, than GD would be that it, despite the (expected) increased computing time per step, would converge with few enough steps as to still beat GD. 

In choosing a second order method, the simplest choice would be a Riemannian version of Newton's method (NM). Despite it's simple design, NM is known to be quite unstable, and need to be sufficiently close to a local minima to guarantee convergence \cite[p.~122]{Boumal2023}. The trust-region method (TR) addresses several of the problems with NM, while still having comparatively fast convergence properties as NM locally. 

Following \cite[p.~131]{Boumal2023}, the goal of each step $k$ of TR is for a point $p_{k}$, is to approximate the pullback $f\circ \mathcal{R}_{p_{k}}(X)$ through an approximation of $T_{p_{k}}\mathrm{SpSt}(2n, 2k)$,
%
\begin{equation*}
f(\mathcal{R}_{p_{k}}(X))\approx m_{p_{k}}(X)=f(p_{k})+\langle \operatorname{grad}f(p_{k}),X \rangle _{p_{k}}+ \frac{1}{2}\langle H_{p_{k}}(X),X \rangle _{p_{k}}.
\end{equation*}
%
$H_{p_{k}}$ can be any self-adjoint linear map on $T_{p_{k}}\mathrm{SpSt}(2n, 2k)$, but in our experiments $H_{p_{k}}$ will be either $\operatorname{Hess}f(p_{k})[X]$ as in \eqref{eq:riemannian_hessian}, or $\operatorname{Proj}_{p_{k}}(\operatorname{D}\overline{\operatorname{grad}}f(p_{k})[X])$ as in \eqref{eq:approximate_hessian}. From \cite[Prop.~5.44]{Boumal2023} the convergence rate of this method is essentially the same as $\operatorname{Hess}f(p)[X]$. 

To choose a step size for TR we demand that the step must reduce the value of $m_{p_{k}}(X)$. Since our model is an approximation of the tangent space, we construct a trust region which is the region around $p_{k}$ where we assume that the error in our approximation is negligible. We solve the TR subproblem
%
\begin{equation}\label{eq:TR_subproblem}
\operatorname*{min}_{X\in T_{p_{k}}\mathrm{SpSt}(2n, 2k)}m_{k}(X)\quad \text{subject to}\quad \lvert \lvert X \rvert  \rvert _{p_{k}}\leq \Delta_{k}
\end{equation}
%
to find the candidate step $X_{k}$,  where candidate for next iterate then is $\hat{p}_{k}=\mathcal{R}_{p_{k}}(X_{k})$. $\Delta_{k}$ denotes the radius of the trust region at that iterate. Depending on how the new step performs (see line 5 of Algorithm \ref{alg:Trust_region}) it is either accepted or rejected. Finally the trust region radius is evaluated to see if it needs to be modified. The procedure is codified in Algorithm \ref{alg:Trust_region}, which is adapted from \cite[Algorithm 3.3]{Boumal2023}. To solve the subproblem in line 2 of Algorithm \ref{alg:Trust_region} we employ, as Jensen and Zimmermann did, the \textit{truncated conjugate gradients method}. We will not go into more detail in this report, but further reading can be found in \cite[p.~131]{Boumal2023}.

% Source boumal. Algorithm 6.3 
\begin{algorithm}[H]
    \caption{Riemannian Trust-region method}\label{alg:Trust_region}
    \textbf{Input:} Initial point $p_{0}\in \mathrm{SpSt}(2n, 2k)$, objective function $f\colon\mathrm{SpSt}(2n, 2k)\to \mathbb{R}$, retraction $\mathcal{R}$, maximum number of iterations $N\in \mathbb{N}$,  maximal radius $\overline{\Delta}>0$, initial radius $\Delta_{0}\in(0,\overline{\Delta})$, ratio of model improvement threshold $\gamma_{\text{min}}>0$, tolerance parameter $\epsilon>0$, Riemannian metric $\langle \cdot,\cdot \rangle_{p}$, with Riemannian gradient $\operatorname{grad}f(p)$, and $\lvert \lvert \cdot,\cdot \rvert  \rvert_{\text{F}}$ denotes the Frobenius norm.
    \begin{algorithmic}[1]
        \For{$0\leq k\leq N$}
        \State Find $X_{k}$ through solving \eqref{eq:TR_subproblem}
        \State $\hat{p}=\mathcal{R}_{p_{k}}(X_{k})$
        \State $\gamma_{k}=\big(f(p_{k})-f(\hat{p})\big)/\big(m_{k}(0)-m_{k}(X_{k})\big)$
        \State Compute new iterate\begin{equation*}
            p_{k+1}=
            \begin{cases}
            \hat{p} & \text{if }\gamma_{k}>\gamma_{\text{min}} \\
            p_{k}  & \text{otherwise}
            \end{cases}
            \end{equation*}
        \State Compute new trust-region radius\begin{equation*}
    \Delta_{k+1}=
    \begin{cases}
    \frac{1}{4}\Delta_{k} & \text{if }\gamma_{k}<\frac{1}{4} \\
    \operatorname{min}\begin{Bmatrix}2\Delta_{k},\overline{\Delta}_{k}\end{Bmatrix} & \text{if }\gamma>\frac{3}{4}\text{ and }\lvert \lvert X_{k} \rvert  \rvert =\Delta_{k} \\
    \Delta_{k} & \text{otherwise}
    \end{cases}
    \end{equation*}  
        \If{$\lvert \lvert \operatorname{grad}f(p_{k}) \rvert  \rvert_{\text{F}}<\epsilon$}
        \State Break
        \EndIf
    \EndFor
    \end{algorithmic}
    \textbf{Output:} Iterates $\{p_{k}\}$
\end{algorithm}

Convergence and stability of TR is presented in-depth in \cite[p.~147]{Boumal2023}. However, for consistency we note here that given sufficient assumptions (which are met in our examples), by \cite[Cor.~6.24]{Boumal2023} for $\left\{ p_{k} \right\}$ generated by TR,
%
\begin{equation*}
\operatorname*{lim~inf}_{ k \to \infty }\lvert \lvert \operatorname{grad}f(p_{k}) \rvert  \rvert _{p_{k}}=0.
\end{equation*}
%
In other words, this means that for all $\epsilon>0$ and $K$ there exists $k\geq K$ such that $\lvert \lvert \operatorname{grad}f(p_{k}) \rvert \rvert_{p_{k}}\leq\epsilon$. 

Now that we have defined the Riemannian counterparts for GD and TR, we have all of the tools necessary to conduct our experiments. In the next section we will discuss some minor details around the implementation before moving on to the actual experiments.